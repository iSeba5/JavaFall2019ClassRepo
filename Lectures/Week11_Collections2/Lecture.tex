\documentclass[12pt]{article}
\usepackage[breaklinks=true]{hyperref}
\usepackage[margin=1in]{geometry}

\usepackage{color}

\definecolor{pblue}{rgb}{0.13,0.13,1}
\definecolor{pgreen}{rgb}{0,0.5,0}
\definecolor{pred}{rgb}{0.9,0,0}
\definecolor{pgrey}{rgb}{0.46,0.45,0.48}

\usepackage{listings}
\lstset{language=Java,
  showspaces=false,
  showtabs=false,
  tabsize=2,
  breaklines=true,
  showstringspaces=false,
  breakatwhitespace=true,
  commentstyle=\color{pgreen},
  keywordstyle=\color{pblue},
  stringstyle=\color{pred},
  basicstyle=\ttfamily,
  frame=single,
  moredelim=[il][\textcolor{pgrey}]{$$},
  moredelim=[is][\textcolor{pgrey}]{\%\%}{\%\%}
}

\title{Java Collections}
\author{
	Melvyn Ian Drag
}
\date{\today}


\begin{document}
\maketitle

\begin{abstract}
$java.util$ provides many containers. These containers are widely used in Java programming and are implementations of the great data structures you hear about in data structures \& algorithms classes. In today's lecture we'll have a look at a few of them and consider when we would want to use them.
\end{abstract}

\section{Sets}
We are going to learn about sets today.
\begin{enumerate}
\item HashSet
\item TreeSet
\end{enumerate}

\section{What Is a HashSet?}

\section{What is a TreeSet?}

\section{Timing Comparison}

\end{document}
